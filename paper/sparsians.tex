%
% File naaclhlt2018.tex
%
%% Based on the style files for NAACL-HLT 2018, which were
%% Based on the style files for ACL-2015, with some improvements
%%  taken from the NAACL-2016 style
%% Based on the style files for ACL-2014, which were, in turn,
%% based on ACL-2013, ACL-2012, ACL-2011, ACL-2010, ACL-IJCNLP-2009,
%% EACL-2009, IJCNLP-2008...
%% Based on the style files for EACL 2006 by 
%%e.agirre@ehu.es or Sergi.Balari@uab.es
%% and that of ACL 08 by Joakim Nivre and Noah Smith

\documentclass[11pt,a4paper]{article}
\usepackage[hyperref]{naaclhlt2018}
\usepackage{times}
\usepackage{latexsym}

\usepackage{amsmath}
\usepackage{amssymb}
\usepackage{url}

\aclfinalcopy % Uncomment this line for all SemEval submissions

%\setlength\titlebox{5cm}
% You can expand the titlebox if you need extra space
% to show all the authors. Please do not make the titlebox
% smaller than 5cm (the original size); we will check this
% in the camera-ready version and ask you to change it back.

\newcommand\BibTeX{B{\sc ib}\TeX}

%Title format for system description papers by task participants
\title{[SystemName] at SemEval-2018 Task 9: Employing Sparse Word Representations for Hypernym Discovery}
%Title format for task description papers by task organizers
%\title{SemEval-2018 Task [TaskNumber]:  [Task Name]}





\author{First Author \\
  Affiliation / Address line 1 \\
  Affiliation / Address line 2 \\
  Affiliation / Address line 3 \\
  {\tt email@domain} \\\And
  Second Author \\
  Affiliation / Address line 1 \\
  Affiliation / Address line 2 \\
  Affiliation / Address line 3 \\
  {\tt email@domain} \\}

\date{}

\begin{document}
\maketitle
\begin{abstract}
  This document contains the instructions for preparing a submission or camera-ready
  manuscript for the proceedings of Semeval-2018. The document itself
  conforms to its own specifications, and is therefore an example of
  what your manuscript should look like. These instructions should be
  used for both papers submitted for review and for final versions of
  accepted papers.  Authors are asked to conform to all the directions
  reported in this document.
\end{abstract}

\section{Introduction}



\section{Introduction}


For further details of the shared task see
\cite{semeval2018task9}.

Our source code in available at \url{https://github.com/begab/fca_hypernymy}.

%\section{Related work} bele fog ez férni 4 oldalba? ha kevés lesz a hely, akkor ezen lehet fogni

\section{Our methodology}
We used the popular skip-gram (SG) and continuous-bag-of-words (CBOW) approaches \cite{DBLP:journals/corr/abs-1301-3781} to train $d=100$ dimensional dense distributed word representations for each sub-corpora. For some subcorpus $x$, we denote the embedding matrix as $W_x \in \mathbb{R}^{\lvert V_x \rvert \times d}$ with $\lvert V_x \rvert$ denoting the size of the vocabulary and $d$ is set to 100 as stated previously.

As a subsequent step we turn the dense vectorial word representations into sparse word vectors akin to \citet{TACL1063} by solving for
\begin{equation}
\min\limits_{D \in \mathcal{C}, \alpha \in \mathbb{R}_{\geq0}} \lVert D\alpha - W \rVert_F + \lambda \lVert \alpha \rVert_1,
\label{nonneg_SPAMS_objective}
\end{equation}
where $\mathcal{C}$ is the convex set of matrices containing only such vectors whose norm does not exceeds $1$ and $\alpha$ contains the sparse coefficients encoding which basis vectors from $D$ takes part in the reconstruction of the vectorial representation of some element of the vocabulary. The only difference compared to \cite{TACL1063} is that here we ensure a non-negativity constraint over the elements of $\alpha$.

For the elements of the vocabulary we ran the formal concept analysis tool of \citet{2010378} \footnote{\url{www.compsens.uni-tuebingen.de/pub/pages/personals/3/concepts.py}}.

\section{Experimental results}

\section{Conclusion}

\bibliography{semeval2018}
\bibliographystyle{acl_natbib}

\end{document}